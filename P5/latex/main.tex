\documentclass[twoside]{article}
\usepackage{amssymb}
\usepackage{amsthm}
\usepackage{amsmath}
\usepackage{amsfonts}
\usepackage[utf8]{inputenc}
\usepackage[spanish]{babel}
\usepackage{tikz}
\usepackage{centernot}
\usepackage{hyperref}
\usepackage{fancyhdr}
\usepackage{lipsum}
\usepackage{subcaption}
\hypersetup{
    colorlinks,
    citecolor=black,
    filecolor=black,
    linkcolor=black,
    urlcolor=black
}
\usepackage{xurl}
\usepackage[top=1in, bottom=1.5in, left=1in, right=1in]{geometry}
\pagestyle{fancy}
\fancyhead{}
\fancyhead[L]{\leftmark}
\fancyfoot{}
\fancyfoot[C]{\thepage}
\newcommand{\enquote}[1]{``#1''}
\usepackage{float}
\usepackage[parfill]{parskip}
\newcommand{\image}[2]{
\begin{figure}[H]
    \includegraphics[width=#1 cm]{../images/#2.png}
    \centering
\end{figure}
}

\title{Práctica 5 SWAP}
\author{XuSheng Zheng}
\date{}

\begin{document}

\maketitle
\tableofcontents
\newpage

\section{Base de datos MySQL}
Empezaremos creando una base de datos en MySQL en m1:
\image{10}{1}
Probamos crear una nueva tabla e insertar datos:
\image{10}{2}

\subsection{Opciones avanzadas}
Además, podemos modificar la tabla una vez que se ha creado. Por ejemplo, introduciendo un campo más:
\image{10}{3}
Y podemos actualizar la fila que habíamos introducido con un valor para este campo nuevo:
\image{10}{4}
También podemos eliminar dicho campo:
\image{10}{5}

\section{Mysqldump}
En esta sección vamos a replicar la base de datos de m1 en m2. Para ello necesitamos primero desactivar las configuraciones de cortafuegos que realizamos en la práctica anterior. Lo podemos hacer fácilmente ejecutando el siguiente script:
\image{6}{6}
Tras ejecutar el script entramos en la base de datos de m1 para evitar que se actualice la base de datos mientras estamos realizando la copia de seguridad:
\image{10}{7}
Ahora ya podemos realizar la copia:
\image{10}{8}
Tras finalizar la copia tenemos que desbloquear las tablas que habíamos bloqueado anteriormente:
\image{10}{9}
Ahora vamos a copiar a m2 el archivo \textit{.sql} que hemos generado:
\image{10}{10}
Ahora creamos en m2 una base de datos con el mismo nombre:
\image{10}{11}
Volcamos los datos que hemos traído desde m1 en m2 y comprobamos:
\image{10}{12}

\subsection{Opciones avanzadas}
Mysqldump admite una gran variedad de opciones, algunas de las que nos pueden servir son:
\begin{itemize}
    \item \textbf{--databases}: permite copiar varias bases de datos a la vez. Usando esta opción también estamos permitiendo que se crean automáticamente las bases de datos si no existen.
    \item \textbf{--all-databases}: similar al anterior pero con todas las bases de datos.
    \item \textbf{--lock-tables}: bloquea las tablas de las bases de datos durante la copia. Cabe destacar que realiza el bloqueo sobre cada base de datos que se está realizando la copia, luego tablas en distintas bases de datos pueden terminar copiándose en diferentes estados.
    \item \textbf{--lock-all-tables}: bloquea todas las tablas de todas las bases de datos.
    \item \textbf{-v}: nos muestra los detalles del proceso de copia.
\end{itemize}
Para probar estas opciones podemos hacer la siguiente prueba:
\image{10}{13}

\section{Configuración Maestro-Esclavo}
En esta sección, vamos a configurar una replicación maestro-esclavo entre las máquinas m1 y m2 siendo m1 el maestro. Para ello, editamos como root el archivo \textit{/etc/mysql/mysql.conf.d/mysqld.cnf} comentado el parámetro bind-address y especificando el id del servidor y el registro binario:
\image{10}{14}
Guardamos el documento y reiniciamos el servicio con \textbf{sudo service mysql restart}. En m2 realizamos un cambio similar cambiando el id del servidor por 2:
\image{10}{15}
Reiniciamos el servicio en m2 y volvemos a m1 para crear un usuario encargado de realizar la replicación. Puesto que en este caso la versión de MySQL es la 8.0.32, ejecutamos los siguientes órdenes:
\image{10}{16}
Utilizamos la sentencia \textbf{mysql\_native\_password} para evitar posibles errores respecto a la autentificación. Para terminar con la configuración en el maestro, comprobamos:
\image{10}{17}
Pasamos ahora a configurar m2. En primer lugar introducimos los datos del maestro que hemos obtenido con la sentencia anterior y arrancamos el esclavo:
\image{10}{18}
Por último, volvemos al maestro para activar las tablas:
\image{10}{20}
Podemos comprobar el estado del esclavo usamos la orden \textbf{SHOW SLAVE STATUS\textbackslash G}:
\image{10}{21}
Vemos que la variable \textbf{Seconds\_Behind\_Master} es 0, lo que indica que no hay ningún error. Si hubiese algún error, tendremos un número distinto del 0 en la variable \textbf{Last\_IO\_Errno} y en \textbf{Last\_IO\_Error} nos indicaría los detalles del error. 

Procedemos ahora a realizar una pequeña prueba. Introduciremos una fila nueva en la tabla de datos en m1:
\image{10}{22}
Y comprobamos en m2:
\image{10}{23}
En este caso la primera consulta es anterior a la inserción en m1. Podemos ver que los datos se han sincronizado satisfactoriamente. 

\section{Configuración Maestro-Maestro}
Acabamos de configurar las dos máquinas de forma que los cambios que realicemos en m1 se sincronizarán en m2. Sin embargo, si los cambios se realizaran en m2, no se sincronizarían automáticamente en m1. Para resolver este problema, vamos a repetir el proceso anterior con m2 como maestro, y así tener una configuración maestro-maestro.

Puesto que ya hemos configurado el archivo \textit{/etc/mysql/mysql.conf.d/mysqld.cnf}, comenzamos directamente creando el usuario esclavo. Puesto que el usuario esclavo de m1 se ha sincronizado también, vamos a crear el usuario con otro nombre:
\image{10}{24}
Consultamos los datos del maestro:
\image{10}{25}
Ahora pasamos a m1 para configurar el esclavo:
\image{10}{26}
Y comprobamos si ha habido algún error:
\image{10}{27}
Puesto que la variable \textbf{Seconds\_Behind\_Master} es 0, no ha habido ningún error. Ahora podemos probar borrando desde m2 la fila que hemos creado anteriormente:
\image{10}{28}
Consultando desde m1:
\image{10}{29}
Podemos ver que la sincronización se ha llevado a cabo correctamente.

\section{Configurar IPTABLES para puerto 3306}
Para las secciones anteriores, hemos tenido que desactivar las reglas IPTABLES que habíamos configurado en la práctica anterior y durante todo este tiempo, hemos estado expuestos. Para resolver este problema, vamos a configurar IPTABLES para permitir la conexión entre ambas máquinas por el puerto 3306. En primer lugar, añadimos al script que habíamos creado en la práctica anterior las siguientes líneas:
\image{10}{30}
Añadimos las mismas líneas para el script de m2 cambiando el IP de las líneas. Ejecutamos las reglas y si no ha habido ningún error los hacemos persistentes:
\image{10}{31}


\newpage
\section{Bibliografía}
\begin{itemize}
    \item \url{https://linux.die.net/man/1/mysqldump}
    \item \url{https://stackoverflow.com/questions/49194719/authentication-plugin-caching-sha2-password-cannot-be-loaded}
    
\end{itemize}

\end{document}
