\documentclass{article}
\usepackage{amssymb}
\usepackage{amsthm}
\usepackage{amsmath}
\usepackage{amsfonts}
\usepackage[utf8]{inputenc}
\usepackage[spanish]{babel}
\usepackage{tikz}
\usepackage{centernot}
\usepackage{hyperref}
\hypersetup{
    colorlinks,
    citecolor=black,
    filecolor=black,
    linkcolor=black,
    urlcolor=black
}
\newcommand{\enquote}[1]{``#1''}
\usepackage{float}
\usepackage[parfill]{parskip}
\newcommand{\image}[2]{
\begin{figure}[H]
    \includegraphics[width=#1 cm]{../images/#2.png}
    \centering
\end{figure}
}

\title{Práctica 1 SWAP}
\author{XuSheng Zheng}
\date{}

\begin{document}

\maketitle
\tableofcontents
\newpage

\section{Instalación de máquinas virtuales}
Comenzamos con la instalación de las máquinas virtuales, puesto que vamos a realizar el mismo procedimiento en ambas máquinas, vamos a centrarnos en la configuración de una de ellas:
\image{8}{1}
Le damos 4GB de RAM y 10 GB de disco duro:
\image{8}{2}
\image{8}{3}
Vamos a instalar Ubuntu Server, en mi caso, será la 20.04:
\image{8}{4}
Arrancamos e introducimos nuestros datos:
\image{10}{5}
Dejamos también que nos instale SSH:
\image{10}{6}

\subsection{Instalación de Apache+PHP+MySQL}
Una vez instalado la máquina virtual, instalamos Apache, PHP y MySQL:
\image{10}{7}
Una vez finalizada la instalación comprobamos:
\image{10}{8}
Podemos ver que el servidor está inactivo, lo activamos:
\image{10}{9}
De mismo modo con MySQL:
\image{10}{10}

Activamos el adaptador NAT y solo-anfitrión:
\image{8}{11}
\image{8}{12}
Podemos usar el comando \textbf{ipconfig} en el anfitrión para ver la puerta de enlace:
\image{8}{13}

\end{document}
